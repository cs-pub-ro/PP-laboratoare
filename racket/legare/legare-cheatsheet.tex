%%%%%%%%%%%%%%%%%%%%%%%%%%%%%%%%%%%%%%%%%%%%%%%%%%%%%%%
% MatPlotLib and Random Cheat Sheet
%
% Edited by Michelle Cristina de Sousa Baltazar
%
% http://matplotlib.org/api/pyplot_summary.html
% http://matplotlib.org/users/pyplot_tutorial.html
%
%%%%%%%%%%%%%%%%%%%%%%%%%%%%%%%%%%%%%%%%%%%%%%%%%%%%%%%

\documentclass[a4paper]{article}
\usepackage[landscape]{geometry}
\usepackage{url}
\usepackage{multicol}
\usepackage{amsmath}
\usepackage{amsfonts}
\usepackage{tikz}
\usetikzlibrary{decorations.pathmorphing}
\usepackage{amsmath,amssymb}
\usepackage{hyperref}

\usepackage{colortbl}
\usepackage{xcolor}
\usepackage{mathtools}
\usepackage{amsmath,amssymb}
\usepackage{enumitem}

% Mihnea
\usepackage{textcomp} % \textquotesingle: Racket: '(1 2 3)
\usepackage{couriers}
\usepackage{listings}
\lstset{
	numbers			= left,
	numberstyle		= \tiny,
    numbersep       = 5pt,
	captionpos		= b,
	breaklines		= true,
	basicstyle		= \ttfamily\footnotesize, 
	tabsize			= 4,
	escapeinside	= {~}{~},
}
\lstdefinelanguage{Racket}{
  morekeywords=[1]{define, define-syntax, define-macro, lambda, define-stream, stream-lambda},
  morekeywords=[2]{begin, call-with-current-continuation, call/cc,
    call-with-input-file, call-with-output-file, case, cond,
    do, else, for-each, if,
    let*, let, let-syntax, letrec, letrec-syntax,
    let-values, let*-values,
    and, or, not, delay, force,
    quasiquote, quote, unquote, unquote-splicing,
    map, fold, syntax, syntax-rules, eval, environment },
  morekeywords=[3]{import, export},
  alsodigit=!\$\%&*+-./:<=>?@^_~,
  sensitive=true,
  morecomment=[l]{;},
  morecomment=[s]{\#|}{|\#},
  morestring=[b]",
  basicstyle=\footnotesize\ttfamily,
  keywordstyle=\color[rgb]{0,.3,.7},
  commentstyle=\color[rgb]{0.133,0.545,0.133},
  stringstyle={\color[rgb]{0.75,0.49,0.07}},
  upquote=true,
  breaklines=true,
  breakatwhitespace=true,
  literate=*{`}{{`}}{1}
}

\title{Racket}
\usepackage[brazilian]{babel}
\usepackage[utf8]{inputenc}

\advance\topmargin-1.0in
\advance\textheight3in
\advance\textwidth3in
\advance\oddsidemargin-1.5in
\advance\evensidemargin-1.5in
\parindent0pt
\parskip1pt
\newcommand{\hr}{\centerline{\rule{3.5in}{1pt}}}
%\colorbox[HTML]{e4e4e4}{\makebox[\textwidth-2\fboxsep][l]{texto}
\begin{document}

\begin{center}{\huge{\textbf{Racket CheatSheet}}}\\
{\large Laborator4}
\end{center}
\begin{multicols*}{3}

\tikzstyle{mybox} = [draw=black, fill=white, very thick,
    rectangle, rounded corners, inner sep=10pt, inner ysep=10pt]
\tikzstyle{fancytitle} =[fill=black, text=white, font=\bfseries]

% Mihnea
\tikzstyle{mybox_code} = [mybox, draw = orange, fill=sandybrown]
\tikzstyle{fancytitle_code} = [fancytitle, fill = orange]

\definecolor{almond}{rgb}{0.94, 0.87, 0.8}
\definecolor{apricot}{rgb}{0.98, 0.81, 0.69}
\definecolor{atomictangerine}{rgb}{1.0, 0.6, 0.4}
\definecolor{sandybrown}{rgb}{0.96, 0.64, 0.38}
\definecolor{buff}{rgb}{0.94, 0.86, 0.51}

\definecolor{persianred}{rgb}{0.8, 0.2, 0.2}
\definecolor{papayawhip}{rgb}{1.0, 0.94, 0.84}
\tikzstyle{mybox_persianred} = [mybox, draw = persianred, fill=papayawhip]
\tikzstyle{fancytitle_persianred} = [fancytitle, fill = persianred]

\definecolor{whitesmoke}{rgb}{0.96, 0.96, 0.96}
\definecolor{wenge}{rgb}{0.39, 0.33, 0.32}
\tikzstyle{mybox_blue} = [mybox, draw = wenge, fill=whitesmoke]
\tikzstyle{fancytitle_blue} = [fancytitle, fill = wenge]

\definecolor{cerise}{rgb}{0.87, 0.19, 0.39}
\definecolor{mistyrose}{rgb}{1.0, 0.89, 0.88}
\tikzstyle{mybox_cerise} = [mybox, draw = cerise, fill=mistyrose]
\tikzstyle{fancytitle_cerise} = [fancytitle, fill = cerise]

\definecolor{pinegreen}{rgb}{0.0, 0.47, 0.44}
\definecolor{bubbles}{rgb}{0.91, 1.0, 1.0}
\tikzstyle{mybox_pinegreen} = [mybox, draw = pinegreen, fill=bubbles]
\tikzstyle{fancytitle_pinegreen} = [fancytitle, fill = pinegreen]

\definecolor{cream}{rgb}{1.0, 0.99, 0.82}
\definecolor{mikadoyellow}{rgb}{1.0, 0.77, 0.05}
\tikzstyle{mybox_mikadoyellow} = [mybox, draw = mikadoyellow, fill=cream]
\tikzstyle{fancytitle_mikadoyellow} = [fancytitle, fill = mikadoyellow]

\definecolor{cornsilk}{rgb}{1.0, 0.97, 0.86}
\tikzstyle{mybox_orange} = [mybox, draw = orange, fill=cornsilk]
\tikzstyle{fancytitle_orange} = [fancytitle, fill = orange]

\definecolor{aliceblue}{rgb}{0.94, 0.97, 1.0}
\definecolor{seagreen}{rgb}{0.18, 0.55, 0.34}
\tikzstyle{mybox_seagreen} = [mybox, draw = seagreen, fill=aliceblue]
\tikzstyle{fancytitle_seagreen} = [fancytitle, fill = seagreen]

\definecolor{jazzberryjam}{rgb}{0.65, 0.04, 0.37}
\definecolor{almond}{rgb}{0.94, 0.87, 0.8}
\tikzstyle{mybox_jazzberryjam} = [mybox, draw = jazzberryjam, fill=almond]
\tikzstyle{fancytitle_jazzberryjam} = [fancytitle, fill = jazzberryjam]

\definecolor{amaranth}{rgb}{0.9, 0.17, 0.31}
\definecolor{bisque}{rgb}{1.0, 0.89, 0.77}
\tikzstyle{mybox_amaranth} = [mybox, draw = amaranth, fill=bisque]
\tikzstyle{fancytitle_amaranth} = [fancytitle, fill = amaranth]

\definecolor{carminered}{rgb}{1.0, 0.0, 0.22}
\definecolor{blanchedalmond}{rgb}{1.0, 0.92, 0.8}
\tikzstyle{mybox_carminered} = [mybox, draw = amaranth, fill=blanchedalmond]
\tikzstyle{fancytitle_carminered} = [fancytitle, fill = carminered]

\definecolor{midnightgreen}{rgb}{0.0, 0.29, 0.33}
\definecolor{lavendermist}{rgb}{0.9, 0.9, 0.98}
\tikzstyle{mybox_midnightgreen} = [mybox, draw = midnightgreen, fill=lavendermist]
\tikzstyle{fancytitle_midnightgreen} = [fancytitle, fill = midnightgreen]

\definecolor{indigo}{rgb}{0.29, 0.0, 0.51}
\definecolor{isabelline}{rgb}{0.96, 0.94, 0.93}
\tikzstyle{mybox_indigo} = [mybox, draw = indigo, fill=isabelline]
\tikzstyle{fancytitle_indigo} = [fancytitle, fill = indigo]

\definecolor{russet}{rgb}{0.5, 0.27, 0.11}
\definecolor{ivory}{rgb}{1.0, 1.0, 0.94}
\tikzstyle{mybox_russet} = [mybox, draw = russet, fill=ivory]
\tikzstyle{fancytitle_russet} = [fancytitle, fill = russet]

\definecolor{neongreen}{rgb}{0.22, 0.88, 0.08}
\definecolor{splashedwhite}{rgb}{1.0, 0.99, 1.0}
\tikzstyle{mybox_neongreen} = [mybox, draw = neongreen, fill=splashedwhite]
\tikzstyle{fancytitle_neongreen} = [fancytitle, fill = neongreen]
%---------------------------------------------------------------------------------

\begin{tikzpicture}
\node [mybox_persianred] (box){%
    \begin{minipage}{0.3\textwidth}
    	{\centering\bf\color{persianred} Colorată \color{black} - zona de vizibilitate pentru id1 \\}
        {\centering\bf Valoare de retur - exprn \\}
		\begin{lstlisting}[language=Racket]
(let ((id1 val1)
      (id2 val2)
      ...
      (idn valn))
   ~\bf\color{persianred}expr1~
   ~\bf\color{persianred}expr2~
   ~\bf\color{persianred}...~
   ~\bf\color{persianred}exprn~)
   
(define a 10)

(let ((a 1) (b (+ a 1)))
   (cons a b))                           (1 . 11)
\end{lstlisting}
    \end{minipage}
};

\node[fancytitle_persianred, right=10pt] at (box.north west) {let};
\end{tikzpicture}

%---------------------------------------------------------------------------------

\begin{tikzpicture}
\node [mybox_seagreen] (box){%
    \begin{minipage}{0.3\textwidth}
    	{\centering\bf\color{seagreen} Colorată \color{black} - zona de vizibilitate pentru id1 \\}
        {\centering\bf Valoare de retur - exprn \\}
		\begin{lstlisting}[language=Racket]
(let* ((id1 val1)
       (id2 ~\bf\color{seagreen}val2~)
       ~\bf\color{seagreen}...~
       (idn ~\bf\color{seagreen}valn~))
   ~\bf\color{seagreen}expr1~
   ~\bf\color{seagreen}expr2~
   ~\bf\color{seagreen}...~
   ~\bf\color{seagreen}exprn~)
   
(define a 10)

(let* ((a 1) (b (+ a 1)))
   (cons a b))                            (1 . 2)
        \end{lstlisting}
    \end{minipage}
};

\node[fancytitle_seagreen, right=10pt] at (box.north west) {let*};
\end{tikzpicture}

%---------------------------------------------------------------------------------

\begin{tikzpicture}
\node [mybox_indigo] (box){%
    \begin{minipage}{0.3\textwidth}
{\centering\bf\small\color{indigo}        	{\centering\bf\color{indigo} nume \color{black} - apare în \color{indigo}corp \color{black}ca un apel recursiv al funcției cu parametrii \color{indigo}id1 .. idn \color{black}și corpul \color{indigo}corp \\}
}
		\begin{lstlisting}[language=Racket]    
(let nume ((id1 val1)
           (id2 val2)
           ...
           (idn valn))
   corp)
   
(let loop ((n 5)
           (fact 1))
   (if (zero? n)
       fact
       (loop (sub1 n) (* n fact))))          120
        \end{lstlisting}
    \end{minipage}
};

\node[fancytitle_indigo, right=10pt] at (box.north west) {named let};
\end{tikzpicture}
%---------------------------------------------------------------------------------

\begin{tikzpicture}
\node [mybox_cerise] (box){%
    \begin{minipage}{0.3\textwidth}
    	{\centering\bf\color{cerise} Colorată \color{black} - zona de vizibilitate pentru id2 \\}
        {\centering\bf Valoare de retur - exprn \\}
		\begin{lstlisting}[language=Racket]
(letrec ((id1 ~\bf\color{cerise}val1~)
         (id2 ~\bf\color{cerise}val2~)
         ~\bf\color{cerise}...~
         (idn ~\bf\color{cerise}valn~))
   ~\bf\color{cerise}expr1~
   ~\bf\color{cerise}expr2~
   ~\bf\color{cerise}...~
   ~\bf\color{cerise}exprn~)
   
;; cand evaluez b, b trebuie sa fi fost definit
(letrec ((a b) (b 1))      
   (cons a b))                             eroare

;; corpul unei inchideri functionale 
;; nu se evalueaza la momentul definirii
(letrec 
     ((even-length? 
       (lambda (L)                   
         (if (null? L)               
             #t                      
             (odd-length? (cdr L)))))
      (odd-length? 
       (lambda (L) 
         (if (null? L)
             #f
             (even-length? (cdr L))))))
   (even-length? '(1 2 3 4 5 6)))              #t
           \end{lstlisting}
    \end{minipage}
};

\node[fancytitle_cerise, right=10pt] at (box.north west) {letrec};
\end{tikzpicture}


%---------------------------------------------------------------------------------

\begin{tikzpicture}
\node [mybox_russet] (box){%
    \begin{minipage}{0.3\textwidth}
    {\centering\bf\ Ca let, pentru expresii care întorc valori multiple\\}
		\begin{lstlisting}[language=Racket]
;; val-expr este o expresie care intoarce n valori
(let-values ( ((id1 id2 .. idn) val-expr)
              ... )
   corp)
   
(let-values (((x y) (quotient/remainder 10 3)))
   (cons x y))                             (3 . 1)
        \end{lstlisting}
    \end{minipage}
};

\node[fancytitle_russet, right=10pt] at (box.north west) {let-values};
\end{tikzpicture}

%---------------------------------------------------------------------------------

\begin{tikzpicture}
\node [mybox_orange] (box){%
    \begin{minipage}{0.3\textwidth}\small
    {\centering\bf\color{orange} sort remove assoc andmap findf splitf-at \\}
		\begin{lstlisting}[language=Racket]
(sort '(5 2 1 6 4) >)                  (6 5 4 2 1)
(remove 2 '(1 2 3 4 3 2 1))          (1 3 4 3 2 1)
(assoc 3 '((1 2) (3 4) (3 6) (4 5)))         (3 4)
(andmap positive? '(1 2 3))                     #t
(andmap number? '(1 b 3))                       #f
(findf (lambda (x) (> x 4)) '(1 3 5 6 4))        5
(findf (lambda (x) (> x 6)) '(1 3 5 6 4))       #f
(splitf-at '(1 3 4 5 6) odd?)                (1 3)
                                           (4 5 6)
(splitf-at '(1 3 4 5 6) even?)                  ()
                                       (1 3 4 5 6)
                              
\end{lstlisting}
    \end{minipage}
};
\node[fancytitle_orange, right=10pt] at (box.north west) {Alte funcții};
\end{tikzpicture}


%---------------------------------------------------------------------------------

\begin{tikzpicture}
\node [mybox_neongreen] (box){%
    \begin{minipage}{0.3\textwidth}\centering
\href{http://docs.racket-lang.org/}{http://docs.racket-lang.org/}
    \end{minipage}
};

\node[fancytitle_neongreen, right=10pt] at (box.north west) {Folositi cu incredere!};
\end{tikzpicture}

%---------------------------------------------------------------------------------

\end{multicols*}
\end{document}
Contact GitHub API Training Shop Blog About
© 2016 GitHub, Inc. Terms Privacy Security Status Help