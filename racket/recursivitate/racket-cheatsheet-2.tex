%%%%%%%%%%%%%%%%%%%%%%%%%%%%%%%%%%%%%%%%%%%%%%%%%%%%%%%
% MatPlotLib and Random Cheat Sheet
%
% Edited by Michelle Cristina de Sousa Baltazar
%
% http://matplotlib.org/api/pyplot_summary.html
% http://matplotlib.org/users/pyplot_tutorial.html
%
%%%%%%%%%%%%%%%%%%%%%%%%%%%%%%%%%%%%%%%%%%%%%%%%%%%%%%%

\documentclass[a4paper]{article}
\usepackage[landscape]{geometry}
\usepackage{url}
\usepackage{multicol}
\usepackage{amsmath}
\usepackage{amsfonts}
\usepackage{tikz}
\usetikzlibrary{decorations.pathmorphing}
\usepackage{amsmath,amssymb}
\usepackage{hyperref}

\usepackage{colortbl}
\usepackage{xcolor}
\usepackage{mathtools}
\usepackage{amsmath,amssymb}
\usepackage{enumitem}

% Mihnea
\usepackage{textcomp} % \textquotesingle: Racket: '(1 2 3)
\usepackage{couriers}
\usepackage{listings}
\lstset{
	numbers			= left,
	numberstyle		= \tiny,
    numbersep       = 5pt,
	captionpos		= b,
	breaklines		= true,
	basicstyle		= \ttfamily\footnotesize, 
	tabsize			= 4,
	escapeinside	= {~}{~},
}
\lstdefinelanguage{Racket}{
  morekeywords=[1]{define, define-syntax, define-macro, lambda, define-stream, stream-lambda},
  morekeywords=[2]{begin, call-with-current-continuation, call/cc,
    call-with-input-file, call-with-output-file, case, cond,
    do, else, for-each, if,
    let*, let, let-syntax, letrec, letrec-syntax,
    let-values, let*-values,
    and, or, not, delay, force,
    quasiquote, quote, unquote, unquote-splicing,
    map, fold, syntax, syntax-rules, eval, environment },
  morekeywords=[3]{import, export},
  alsodigit=!\$\%&*+-./:<=>?@^_~,
  sensitive=true,
  morecomment=[l]{;},
  morecomment=[s]{\#|}{|\#},
  morestring=[b]",
  basicstyle=\footnotesize\ttfamily,
  keywordstyle=\color[rgb]{0,.3,.7},
  commentstyle=\color[rgb]{0.133,0.545,0.133},
  stringstyle={\color[rgb]{0.75,0.49,0.07}},
  upquote=true,
  breaklines=true,
  breakatwhitespace=true,
  literate=*{`}{{`}}{1}
}

\title{Racket-2}
\usepackage[brazilian]{babel}
\usepackage[utf8]{inputenc}

\advance\topmargin-1.0in
\advance\textheight3in
\advance\textwidth3in
\advance\oddsidemargin-1.5in
\advance\evensidemargin-1.5in
\parindent0pt
\parskip1pt
\newcommand{\hr}{\centerline{\rule{3.5in}{1pt}}}
%\colorbox[HTML]{e4e4e4}{\makebox[\textwidth-2\fboxsep][l]{texto}
\begin{document}

\begin{center}{\huge{\textbf{Racket CheatSheet}}}\\
{\large Laborator 2}
\end{center}
\begin{multicols*}{3}

\tikzstyle{mybox} = [draw=black, fill=white, very thick,
    rectangle, rounded corners, inner sep=10pt, inner ysep=10pt]
\tikzstyle{fancytitle} =[fill=black, text=white, font=\bfseries]

% Mihnea
\tikzstyle{mybox_code} = [mybox, draw = orange, fill=sandybrown]
\tikzstyle{fancytitle_code} = [fancytitle, fill = orange]

\definecolor{almond}{rgb}{0.94, 0.87, 0.8}
\definecolor{apricot}{rgb}{0.98, 0.81, 0.69}
\definecolor{atomictangerine}{rgb}{1.0, 0.6, 0.4}
\definecolor{sandybrown}{rgb}{0.96, 0.64, 0.38}
\definecolor{buff}{rgb}{0.94, 0.86, 0.51}

\definecolor{persianred}{rgb}{0.8, 0.2, 0.2}
\definecolor{papayawhip}{rgb}{1.0, 0.94, 0.84}
\tikzstyle{mybox_persianred} = [mybox, draw = persianred, fill=papayawhip]
\tikzstyle{fancytitle_persianred} = [fancytitle, fill = persianred]

\definecolor{whitesmoke}{rgb}{0.96, 0.96, 0.96}
\definecolor{wenge}{rgb}{0.39, 0.33, 0.32}
\tikzstyle{mybox_blue} = [mybox, draw = wenge, fill=whitesmoke]
\tikzstyle{fancytitle_blue} = [fancytitle, fill = wenge]

\definecolor{cerise}{rgb}{0.87, 0.19, 0.39}
\definecolor{mistyrose}{rgb}{1.0, 0.89, 0.88}
\tikzstyle{mybox_cerise} = [mybox, draw = cerise, fill=mistyrose]
\tikzstyle{fancytitle_cerise} = [fancytitle, fill = cerise]

\definecolor{pinegreen}{rgb}{0.0, 0.47, 0.44}
\definecolor{bubbles}{rgb}{0.91, 1.0, 1.0}
\tikzstyle{mybox_pinegreen} = [mybox, draw = pinegreen, fill=bubbles]
\tikzstyle{fancytitle_pinegreen} = [fancytitle, fill = pinegreen]

\definecolor{cream}{rgb}{1.0, 0.99, 0.82}
\definecolor{mikadoyellow}{rgb}{1.0, 0.77, 0.05}
\tikzstyle{mybox_mikadoyellow} = [mybox, draw = mikadoyellow, fill=cream]
\tikzstyle{fancytitle_mikadoyellow} = [fancytitle, fill = mikadoyellow]

\definecolor{cornsilk}{rgb}{1.0, 0.97, 0.86}
\tikzstyle{mybox_orange} = [mybox, draw = orange, fill=cornsilk]
\tikzstyle{fancytitle_orange} = [fancytitle, fill = orange]

\definecolor{aliceblue}{rgb}{0.94, 0.97, 1.0}
\definecolor{seagreen}{rgb}{0.18, 0.55, 0.34}
\tikzstyle{mybox_seagreen} = [mybox, draw = seagreen, fill=aliceblue]
\tikzstyle{fancytitle_seagreen} = [fancytitle, fill = seagreen]

\definecolor{jazzberryjam}{rgb}{0.65, 0.04, 0.37}
\definecolor{almond}{rgb}{0.94, 0.87, 0.8}
\tikzstyle{mybox_jazzberryjam} = [mybox, draw = jazzberryjam, fill=almond]
\tikzstyle{fancytitle_jazzberryjam} = [fancytitle, fill = jazzberryjam]

\definecolor{amaranth}{rgb}{0.9, 0.17, 0.31}
\definecolor{bisque}{rgb}{1.0, 0.89, 0.77}
\tikzstyle{mybox_amaranth} = [mybox, draw = amaranth, fill=bisque]
\tikzstyle{fancytitle_amaranth} = [fancytitle, fill = amaranth]

\definecolor{carminered}{rgb}{1.0, 0.0, 0.22}
\definecolor{blanchedalmond}{rgb}{1.0, 0.92, 0.8}
\tikzstyle{mybox_carminered} = [mybox, draw = amaranth, fill=blanchedalmond]
\tikzstyle{fancytitle_carminered} = [fancytitle, fill = carminered]

\definecolor{midnightgreen}{rgb}{0.0, 0.29, 0.33}
\definecolor{lavendermist}{rgb}{0.9, 0.9, 0.98}
\tikzstyle{mybox_midnightgreen} = [mybox, draw = midnightgreen, fill=lavendermist]
\tikzstyle{fancytitle_midnightgreen} = [fancytitle, fill = midnightgreen]

\definecolor{indigo}{rgb}{0.29, 0.0, 0.51}
\definecolor{isabelline}{rgb}{0.96, 0.94, 0.93}
\tikzstyle{mybox_indigo} = [mybox, draw = indigo, fill=isabelline]
\tikzstyle{fancytitle_indigo} = [fancytitle, fill = indigo]

\definecolor{russet}{rgb}{0.5, 0.27, 0.11}
\definecolor{ivory}{rgb}{1.0, 1.0, 0.94}
\tikzstyle{mybox_russet} = [mybox, draw = russet, fill=ivory]
\tikzstyle{fancytitle_russet} = [fancytitle, fill = russet]

\definecolor{neongreen}{rgb}{0.22, 0.88, 0.08}
\definecolor{splashedwhite}{rgb}{1.0, 0.99, 1.0}
\tikzstyle{mybox_neongreen} = [mybox, draw = neongreen, fill=splashedwhite]
\tikzstyle{fancytitle_neongreen} = [fancytitle, fill = neongreen]

\tikzstyle{mybox_skyblue} = [mybox, draw = blue!60, fill=splashedwhite]
\tikzstyle{fancytitle_skyblue} = [fancytitle, fill = blue!60]



%-----------------------------------------------------------------------------
\begin{tikzpicture}
\node [mybox_cerise] (box){%
    \begin{minipage}{0.3\textwidth}
		\begin{lstlisting}[language=Racket]
; ~suma elementelor unei liste~
(define (sum-list L)  


; ~aici nu avem nevoie de funcție auxiliară~


  (if (null? L)
      0    ; ~la sfârșit creăm valoarea inițială~
      (+ (car L) (sum-list (cdr L))) 
     ; ^ ~construim rezultatul pe revenire~
     ;  ~(după întoarcerea din recursivitate)~
      ))
; ~fiecare apel recursiv întoarce rezultatul corespunzător argumentelor~
      
      
      
      
; ~concatenarea a două liste~
(define (app L1 L2)
  
  
  
  
  
  (if (null? L1)
      L2  ; ~când L1 este vidă, întoarcem L2~
      (cons (car L1) (app (cdr L1) L2))
      ; ^ ~construim rezultatul pe revenire
      ))
        \end{lstlisting}
        \begin{itemize}\itemsep.2ex
        	\item fiecare apel recursiv se pune pe stivă
            \item complexitate spațială O(n)
        	\item scriere mai simplă
        \end{itemize}
    \end{minipage}
};
\node[fancytitle_cerise, right=10pt] at (box.north west) {Recursivitate pe stivă};
\end{tikzpicture}

\vskip50ex % pentru a trece la următoarea coloană

%-----------------------------------------------------------------------------
\begin{tikzpicture}
\node [mybox_seagreen] (box){%
    \begin{minipage}{0.3\textwidth}
		\begin{lstlisting}[language=Racket]
; ~suma elementelor unei liste~
(define (sum-list L)
  (sum-list-tail 0 L))  ; <-- ~funcție ajutătoare~
               ; ^ ~valoarea inițială pentru sumă~

                 ; ~în sum construim rezultatul~
(define (sum-list-tail sum L)
  (if (null? L)
      sum        ; ~la sfârșit avem rezultatul gata~
      (sum-list-tail
        (+ sum (car L))
       ; ^ ~construim rezultatul pe avans~
       ;  ~(pe măsură ce intrăm în recursivitate)~
        (cdr L))))
  ; ~funcția întoarce direct rezultatul apelului recursiv -- toate apelurile recursive întorc același rezultat, pe cel final~


; ~concatenarea a două liste~
(define (app A B)
  (app-iter B (reverse A)))
  ; ~nevoie de funcție ajutătoare~
  ; ~rezultatul este construit în ordine inversă~

(define (app-iter B Result)
  (if (null? B) ; ~la sfârșit rezultatul e complet~
      (reverse Result) ; ~inversăm rezultatul~
      (app-iter (cdr B) (cons (car B) Result))))
		; ~construim rezultatul pe avans~
		\end{lstlisting}
        
        \begin{itemize}\itemsep.5ex
        	\item apelurile recursive nu consumă spațiu pe stivă -- execuția este optimizată știind că rezultatul apelului recursiv este întors direct, fără operații suplimentare.
            \item complexitatea spațială este dată doar de spațiul necesar pentru acumulator -- de exemplu la \texttt{sum-list-tail} complexitatea spațială este O(1).
            \item scriere mai complexă, necesită de multe ori funcție auxiliară pentru a avea un parametru suplimentar pentru construcția rezultatului (rol de acumulator), mai ales dacă tipul natural de recursivitate al funcției este pe stivă.
            \vspace*{-1.5ex}
            \begin{itemize}
            	\item \textbf{Atenție:} uneori, rolul acumulatorului poate fi preluat de unul dintre parametri, caz în care nu este nevoie nici de funcția suplimentară.
            \end{itemize}
            \item rezultatul este construit în ordine inversă
        \end{itemize}
    \end{minipage}
};
\node[fancytitle_seagreen, right=10pt] at (box.north west) {Recursivitate pe coadă};
\end{tikzpicture}

\vskip50ex % pentru a trece la următoarea coloană











%-----------------------------------------------------------------------------
\begin{tikzpicture}
\node [mybox_persianred] (box){%
    \begin{minipage}{0.3\textwidth}
    	{\centering\bf\color{persianred} (nume\_functie arg1 arg2 ...) \\}
		\begin{lstlisting}[language=Racket]
(max 2 3)               3
(+ 2 3)                 5
        \end{lstlisting}
    \end{minipage}
};
\node[fancytitle_persianred, right=10pt] at (box.north west) {Sintaxa Racket};
\end{tikzpicture}

%-----------------------------------------------------------------------------
\begin{tikzpicture}
\node [mybox_persianred] (box){%
    \begin{minipage}{0.3\textwidth}\centering
%     	{\centering\bf\color{persianred}
		\begin{lstlisting}[language=Racket]
DA: (cons x L)        NU: (append (list x) L)
	                  NU: (append (cons x '()) L)
DA: (if c vt vf)      NU: (if (equal? c #t) vt vf)
DA: (null? L)         NU: (= (length L) 0)
DA: (zero? x)         NU: (equal? x 0)
DA: test              NU: (if test #t #f)
DA: (or ceva1 ceva2)  NU: (if ceva1 #t ceva2)
DA: (and ceva1 ceva2) NU: (if ceva1 ceva2 #f)
        \end{lstlisting}
    \end{minipage}
};
\node[fancytitle_persianred, right=10pt] at (box.north west) {AȘA  DA / AȘA NU};
\end{tikzpicture}

%-----------------------------------------------------------------------------
\begin{tikzpicture}
\node [mybox_skyblue] (box){%
    \begin{minipage}{0.3\textwidth}\centering
{\centering\bf\color{blue!60} overlay, image-height, beside, above \\}
\begin{lstlisting}[language=Racket,escapeinside=||]
(overlay |\includegraphics[scale=0.4]{imgs/fig1.png}|) ; => |\includegraphics[scale=0.5]{imgs/fig2.png}|
(image-height (circle 20 "solid" "red")) ; => 40
(image-height |\includegraphics[scale=0.4]{imgs/fig0.png}|) ; => 60
(image-width |\includegraphics[scale=0.4]{imgs/fig0.png}|) ; => 60
(beside |\includegraphics[scale=0.3]{imgs/fig3.png}||\includegraphics[scale=0.3]{imgs/fig4.png}|) ; => |\includegraphics[scale=0.3]{imgs/fig6.png}|
(above |\includegraphics[scale=0.3]{imgs/fig3.png}||\includegraphics[scale=0.3]{imgs/fig4.png}|) ; => |\includegraphics[scale=0.3]{imgs/fig7.png}|
\end{lstlisting}
    \end{minipage}
};
\node[fancytitle_skyblue, right=10pt] at (box.north west) {Imagini în Racket};
\end{tikzpicture}


%-----------------------------------------------------------------------------
\begin{tikzpicture}
\node [mybox_neongreen] (box){%
    \begin{minipage}{0.3\textwidth}\centering
\href{http://docs.racket-lang.org/}{http://docs.racket-lang.org/}
    \end{minipage}
};
\node[fancytitle_neongreen, right=10pt] at (box.north west) {Folosiți cu încredere!};
\end{tikzpicture}


\end{multicols*}
\end{document}
Contact GitHub API Training Shop Blog About
© 2016 GitHub, Inc. Terms Privacy Security Status Help
